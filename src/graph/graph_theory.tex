\subsection{Graph Theory}

\noindent \textbf{트리의 중심} \\
모든 트리는 1개 or 2개의 중심을 가지며 2개라면 그 두 정점은 반드시 인접함 \\
트리의 모든 지름은 반드시 트리의 중심을 지남(=공유함) \\
트리의 지름이 짝수라면: 중심은 1개, 지름의 중간지점이 중심 \\
홀수라면: 중심은 2개, 지름의 가운데 간선으로 연결된 두 정점이 중심 \\
즉, 중심이 1개라면 모든 지름은 그 중심을 지나고 2개라면 모든 지름은 그 두 중심을 잇는 간선을 지남 \\
트리의 모든 지름이 공유하는 정점들의 집합은 항상 하나의 연결된 경로를 이룸

\vspace{5pt}
\noindent \textbf{평면그래프} \\
연결된 평면 그래프에서 $v-e+f=2$ (v:정점, e: 간선, f: 면) \\
$v \ge 3$인 중복 간선없는 단순 평면 그래프는 $e \le 3v-6$을 만족 \\
$2e=$각 면에서 변의 개수의 합$\ge 3f$이므로 $v-e+f=2$와 연립하면 $e \le 3v-6$ \\
이분그래프라면 사이클의 최소 길이가 4이므로 $2e \ge 4f, e \le 2v-4$

\vspace{5pt}
\noindent \textbf{매칭} \\
Gallai's Identity: $|IS_{max}| + |VC_{min}| = |V|$ \\
$IS_{max}$계산은 일반그래프에선 NP-hard \\
Kőnig's theorem: 이분그래프에서 최대매칭=$|VC_{min}|$ \\
$IS_{max,L} = L - VC_{min,L}, IS_{max,R} = R - VC_{min,R}$ \\
Dilworth's theorem: 부분 순서 집합에서 최대 반사슬의 크기는 사슬 분할의 최소 개수와 같음 \\
antichain은 $v_{in}$이 $IS_{max,L}$이고 \&\& $v_{out}$이 $IS_{max, R}$인 v들을 모으면 됨 \\
clique는 complement graph에서의 Independent set과 같음 \\
hall's marriage theorem: 이분그래프 $G=(L+R,E)$에서 $L$의 부분집합 $S$에 대해 이와 연결된 $R$의 부분집합을 $N(S)$라 할 때, 이 이분그래프에 perfect matching이 존재할 필요충분조건은 모든 $S$에 대하여 $|S| \le |N(S)|$ \\
-> k-정규 이분 그래프는 항상 perfect matching을 가짐 \\